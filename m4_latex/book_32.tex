\documentclass[a4paper,fleqn]{book}

%% Language and font encodings
\usepackage[english,russian]{babel}
\usepackage[utf8x]{inputenc}
\usepackage[T2A]{fontenc}
\usepackage{fullpage} 
\textwidth=15cm 
\topmargin=-1cm 
\usepackage{indentfirst}

%% useful packages
\usepackage{amsmath}
\usepackage{amssymb}
\usepackage[colorlinks=true, allcolors=blue]{hyperref}
\usepackage{color}
\definecolor{light-gray}{rgb}{0.8,0.8,0.8}
\setcounter{chapter}{6} 
\setcounter{section}{4} 
\title{}
\date{}
\begin{document}
\pagestyle{empty}
\begin{quote}
\large\textit{6.5 Bon voyage}\hfill \textbf{121} \\

\colorbox{light-gray}{
\begin{minipage}{\textwidth}
\textbf{Problem 6.29 \; Only real roots}\\
Show that all roots of $\tan x-x$ are real.\\\\
\textbf{Promble 6.30 \; Exact Basel sum}\\
Use the polynomial analogy to evaluate the Basel sum\\
\[\sum_{1}^\infty \frac{1}{n^2} \quad\hfill(6.44)\]\\
Compare your result with your solution to Problem 6.22.\\\\
\textbf{Problem 6.31 \; Misleading alternative expansions}\\
Squaring and taking the reciprocal of $\tan x=x$ gives $\cot x^2=x^{-2}$; equivalently,\\
${\cot}^2x-x^{-2}=0$. Therefore,if x is a root of $\tan x-x$, it is a root of ${\cot}^2x-x^{-2}$.\\
The Taylor expansion of ${\cot}^2x-x^{-2}$ is\\
\[-\frac{2}{3}\;\left(1-\frac{1}{10}x^2-\frac{1}{63}x^4-\ldots\right).\hfill(6.45)\]\\
Because the coefficient of $x^2$ is $-1/10$, the tangent-root sum S---for $\cot x=x^{-2}$\\
and therefore $\tan x=x$---should be $1/10$. \;As we found experimentally and\\
analytically for $\tan x=x$, the conclusion is correct.  However, what is wrong\\
with the reasoning?\\\\
\textbf{Problem 6.32 \; Fourth powers of the reciprocals}\\
The Taylor series for $\sin x-x\cos x$ continues\\
\[\frac{x^3}{3}\;\left(1-\frac{x^2}{10}+\frac{x^4}{280}-\ldots\right).\hfill(6.46)\]\\
Therefore find $\sum x_n^{-4}$ for the positive roots of $\tan x=x$. \;Check numerically\\
that your result is plausible.\\\\
\textbf{Problem 6.33 \; Other source equations for the roots}\\
Find $\sum x_n^{-4}$, where the $x_n$ are the positive roots of $\cos x$.
\end{minipage}
}\\\\
\end{quote}
\section{Bon voyage}
\begin{quote}
\Large\begin{minipage}{\textwidth}I hope that you have enjoyed incorporating street-fighting methods into\\
your problem-solving toolbox. May you find diverse opportunities to use\\
dimensional analysis, easy cases, lumping, pictorial reasoning, taking out\\
the big part, and analogy. As you apply the tools, you will sharpen\\
them---—and even build new tools.
\end{minipage}
\end{quote}

\newpage
\mbox{}
\newpage

\begin{quote}
\Huge\textbf{Bibliography}\\\\\\\\\\\\\\\\
\normalsize\pagestyle{empty}
[1]\; P. Agnoli and G. D’Agostini. Why does the meter beat the second?.\\
 \emph{arXiv:physics/0412078v2}, 2005. Accessed 14 September 2009.\newline\newline
[2]\; John Morgan Allman. \emph{Evolving Brains}. W. H. Freeman, New York,             1999.\newline\newline 
[3]\; Gert Almkvist and Bruce Berndt. Gauss, Landen, Ramanujan, the arithmetic-\\geometric
mean, ellipses, $\pi$, and the Ladies Diary.\emph{American Mathematical Monthly},\newline 95(7):585–608, 1988.\newline\newline
[4]\; William J.H.Andrewes(Ed.).\emph{The Quest for Longitude: The Proceedings of the Longi-}\newline
\emph{tude Symposium, Harvard University, Cambridge, Massachusetts, November 4–6, 1993}. 
Collection of Historical Scientific Instruments,  Harvard University, Cambridge,\\
Massachusetts, 1996.\newline\newline
[5]\;  Petr Beckmann. \emph{A History of Pi}. . Golem Press, Boulder, Colo., 4th edition, 1977.\newline\newline
[6]\;  Lennart Berggren, Jonathan Borwein and Peter Borwein (Eds.). Pi, A Source Book.
Springer, New York, 3rd edition, 2004.\newline\newline
[7]\; John Malcolm Blair. \emph{The Control of Oil}. Pantheon Books, New York, 1976.\newline\newline
[8]\; Benjamin S. Bloom. The 2 sigma problem: The search for methods of group
instruction as effective as one-to-one tutoring. \emph{Educational Researcher}, 13(6):4–16,1984.\newline\newline
[9]\; E. Buckingham. On physically similar systems. \emph{. Physical Review},w, 4(4):345–376,
1914.\newline\newline
[10]\; Barry Cipra.\emph{. Misteaks:And How to Find Them Before the Teacher Does}.AK Peters,
Natick, Massachusetts, 3rd edition, 2000.\newline\newline
[11]\; David Corfield. \emph{Towards a Philosophy of Real Mathematics}.Cambridge University
Press, Cambridge, England, 2003.\newline\newline
[12]\; T. E. Faber.\emph{Fluid Dynamics for Physicists}. Cambridge University Press, Cambridge,
England, 1995.\newline\newline
[13]\; L. P. Fulcher and B. F. Davis. Theoretical and experimental study of the motion
of the simple pendulum. \emph{. American Journal of Physics},44(1):51–55, 1976.\newline\newline
[14]\; George Gamow. \emph{Thirty Years that Shook Physics: The Story of Quantum Theory.}
Dover, New York, 1985.\newline\newline
[15]\; Simon Gindikin.\emph{Tales of Mathematicians and Physicists}.Springer, New York, 2007.
\newpage
\large\textbf{124} \hfill\\
\normalsize\pagestyle{empty}\newline\newline
[16]\; Fernand Gobet and Herbert A. Simon. The role of recognition processes and
look-ahead search in time-constrained expert problem solving: Evidence from. grand-master-level chess. \emph{Psychological Science}, 7(1):52-55, 1996.\newline\newline
[17]\; Ronald L. Graham, Donald E. Knuth and Oren Patashnik.\emph{Concrete Mathematics}.
Addison–Wesley, Reading, Massachusetts, 2nd edition, 1994.\newline\newline
[18]\; Godfrey Harold Hardy, J. E. Littlewood and G. Polya.\emph{Inequalities}.Cambridge
University Press, Cambridge, England, 2nd edition, 1988.\newline\newline
[19]\; William James.\emph{The Principles of Psychology}.Harvard University Press, Cambridge,
MA, 1981. Originally published in 1890.\newline\newline
[20]\; Edwin T. Jaynes. Information theory and statistical mechanics. \emph{Physical Review},\\106(4):620–630, 1957.\newline\newline
[21]\; Edwin T. Jaynes.\emph{Probability Theory: The Logic of Science}. Cambridge University
Press, Cambridge, England, 2003.\newline\newline
[22]\; A. J. Jerri. The Shannon sampling theorem—Its various extensions and applica-tions:
A tutorial review.\emph{Proceedings of the IEEE},65(11):1565–1596, 1977.\newline\newline
[23]\; Louis V. King. On some new formulae for the numerical calculation of the mutual
induction of coaxial circles.  \emph{Proceedings of the Royal Society of London. Series A,\\
Containing Papers of a Mathematical and Physical Character}, 100(702):60–66, 1921.\newline\newline
[24]\; Charles Kittel, Walter D. Knight and Malvin A. Ruderman. Mechanics, volume 1
of \emph{The Berkeley Physics Course}. McGraw–Hill, New York, 1965.\newline\newline
[25]\; Anne Marchand. Impunity for multinationals. ATTAC, 11 September 2002.\newline\newline
[26]\; Mars Climate Orbiter Mishap Investigation Board. Phase I report. Technical Re-port,
NASA, 1999.\newline\newline
[27]\; Michael R.Matthews.\emph{Time for Science Education:How Teaching the History and
 Philosophy of Pendulum Motion can Contribute to Science Literacy}.   Kluwer, New
York, 2000.\newline\newline
[28]\; R. D.  Middlebrook.  Low-entropy expressions: the key to design-oriented analy-\\sis.
In \emph{Frontiers in Education Conference, 1991.Twenty-First Annual Conference.‘En-gineering  Education in a New World Order’.  Proceedings},  pages 399–403,  Purdue\\
University, West Lafayette, Indiana, September 21–24, 1991.\newline\newline
[29]\;  R. D. Middlebroo .  Methods of design-oriented analysis:  The quadratic equa-\\tion
revisisted. \emph{In Frontiers in Education, 1992. Proceedings. Twenty-Second Annual
Conference},  pages 95–102, Vanderbilt University, November 11–15, 1992.\newline\newline
[30] Paul J. Nahin. \emph{When Least is Best: How Mathematicians Discovered Many Clever
Ways to Make Things as Small (or as Large) as Possible}.e. Princeton University Press,
Princeton, New Jersey, 2004.\newline\newline
[31]\; ] Roger B.Nelsen. \emph{Proofs without Words: Exercises in Visual Thinking}. Mathematical
Association of America, Washington, DC, 1997.



\end{quote}
\end{document}




